\documentclass[12pt]{article}

\newcommand{\xmlvm}{\texttt{xmlvm} }

\title{XMLVM User Manual}
\date{}

\begin{document}

\maketitle 

\section{Overview}


\section{Invoking XMLVM}

XMLVM can be invoked via the \xmlvm command line tool. Its behavior is
controlled by numerous command line arguments. \xmlvm reads in one or
more source files, processes them according to the command line
options, and then writes out one or more destination files.

\begin{description}

\item[\texttt{--in=$<$path$>$}] The source files are specified via one
  or more \texttt{--in} options. The \texttt{--in} options requires a
  parameter that denotes a directory. \xmlvm will traverse this
  directory recursively and process any \texttt{*.class},
  \texttt{*.exe}, or \texttt{*.xmlvm} files it encounters. It is
  possible to use wildcards to filter out certain files in a
  directory. It is possible to specify multiple \texttt{--in}
  parameters. At least one \texttt{--in} parameter is required.

\item[\texttt{--out=$<$path$>$}] The output generated by \xmlvm is
  written to a directory specified by the \texttt{--out} parameter.
  The argument \texttt{$<$path$>$}has to denote a directory name. It
  the directory does not exist, \xmlvm will create it. If the
  \texttt{--out} parameter is omitted, the current directory is the
  default.

\item[\texttt{--jvm}] The input files are cross-compiled to
  XMLVM$_{JVM}$.

\item[\texttt{--clr}] The input files are cross-compiled to
  XMLVM$_{CLR}$

\item[\texttt{--dfa}] A DFA (Data Flow Analysis) is performed on the
  input files. Currently the DFA will only be performed for
  XMLVM$_{CLR}$ programs. This option cannot be used in conjunction
  with any other code generating option.

\item[\texttt{--class}] The input files are cross-compiled to Java
  class files.

\item[\texttt{--exe}] The input files are cross-compiled to a .NET
  execurable.

\item[\texttt{--js}] The input files are cross-compiled to JavaScript.

\item[\texttt{--cpp}] The input files are cross-compiled to C++.

\item[\texttt{--python}] The input files are cross-compiled to Python.

\item[\texttt{--objc}] The input files are cross-compiled to
  Objective-C.

\item[\texttt{--iphone-app=$<$app\_name$>$}] Cross-compiles an
  application to an iPhone application. The output directory specified
  by \texttt{--out} will contain a ready to compile iPhone
  application. The application will be called
  \texttt{$<$app\_name$>$}. This option implies \texttt{--objc}.

\item[\texttt{--android2iphone=$<$app\_name$>$}] Cross-compiles an
  Android application to the iPhone. The output directory specified by
  \texttt{--out} will contain a ready to compile iPhone application.
  The application will be called \texttt{$<$app\_name$>$}. This option
  implies \texttt{--iphone-app}.

\item[\texttt{--qx-app=$<$app\_name$>$}] Cross-compiles an application
  to a Qooxdoo application. The environment variable
  \texttt{QOOXDOO\_HOME} needs to point to the base directory of the
  Qooxdoo installation. The application will be called
  \texttt{$<$app\_name$>$}. The output directory specified by
  \texttt{--out} will contain a ready to run Qooxdoo application. This
  option implies \texttt{--js} and requires option \texttt{--qx-main}.

\item[\texttt{--qx-main=$<$main-class$>$}] This option denotes the
  entry point of the generated Qooxdoo application. It requires a full
  qualified name as a parameter. This option can only be used in
  conjunction with option \texttt{--qx-app}.

\item[\texttt{--qx-debug}] Creates a debug version of the Qooxdoo
  application.  If not specified, a ready-to-deploy version will be
  generated.  Requires option \texttt{--qx-app}.

\item[\texttt{--version}] Prints the version of XMLVM.

\item[\texttt{--quiet}] No diagnostic messages are printed.

\end{description}

\section{Examples}

\begin{description}

\item[\texttt{xmlvm --in=/foo/bar}] $ $

  The directory \texttt{/foo/bar} is searched recursively for
  \texttt{*.class}, \texttt{*.exe}, and \texttt{*.xmlvm} files. For
  \texttt{*.class} files, XMLVM$_{JVM}$ is generated. For
  \texttt{*.exe} files, XMLVM$_{CLR}$ is generated. Files with suffix
  \texttt{*.xmlvm} are copied to the output directory. Other files
  with different suffices are ignored. Since no \texttt{--out}
  parameter was given, the default output directory is ``.'' (the
  current directory).

\item[\texttt{xmlvm --in=/foo/*.class --in=/bar/*.exe --out=/bin}] $ $

  The directory \texttt{/foo} is searched recursively for
  \texttt{*.class} and the directory \texttt{/bar} is searched
  recursively for \texttt{*.exe} files. Files with other suffices are
  ignored. For \texttt{*.class} files, XMLVM$_{JVM}$ is generated. For
  \texttt{*.exe} files, XMLVM$_{CLR}$ is generated. The resulting
  \texttt{*.xmlvm} files are placed in directory \texttt{/bin}.

\item[\texttt{xmlvm --in=/foo --jvm}] $ $

  The directory \texttt{/foo} is searched recursively for
  \texttt{*.class}, \texttt{*.exe}, and \texttt{*.xmlvm} files. In all
  cases, the generated output will always be XMLVM$_{JVM}$. For
  \texttt{*.exe} files as well as \texttt{*.xmlvm} files containing
  something other than XMLVM$_{JVM}$ will be cross-compiled
  XMLVM$_{JVM}$.

\item[\texttt{xmlvm --in=/foo --class}] $ $

  Same as the previous example, however instead of generating
  XMLVM$_{JVM}$ files, Java \texttt{*.class} files that can be
  executed by a Java virtual machine will be generated.

\item[\texttt{xmlvm --in=/foo --iphone-app=TheApplication}] $ $

  Same as the previous example, however instead of creating Java
  \texttt{*.class} files, an iPhone application will be generated. The
  output directory will contain the ready to compile Objective-C
  source code including all necessary auxiliary files such as
  \texttt{Info.plist} and a \texttt{Makefile}. The iPhone application
  will be called \texttt{TheApplication} using a default icon.

\item[\texttt{xmlvm --in=/foo --android2iphone=TheApplication}] $ $

  Same as the previous example, but will also copy the Android
  compatibility libraries to the output directory. This effectively
  allows Java-based Android applications to be cross-compiled to the
  iPhone.

\item[\texttt{xmlvm --in=/foo --qx-app=TheApplication
    --qx-main=com.acme.Main}] $ $

  The directory \texttt{/foo} is searched recursively for
  \texttt{*.class}, \texttt{*.exe}, and \texttt{*.xmlvm} files. All
  files will be cross-compiled to JavaScript. With the help of the
  Qooxdoo build scripts, the output directory will contain a ready to
  be deployed AJAX application. The main entry point of the
  application is \texttt{com.acme.Main}.

\end{description}

\bibliographystyle{plain}

\end{document}
